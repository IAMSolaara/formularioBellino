\documentclass[../../main]{subfiles}
\begin{document}
\subsection{Funzioni Logaritmiche}
\begin{center}
    \noindent\makebox[\textwidth]{
        \label{tab:derivate:logaritmiche}
        \begin{tabular}{ |p{5em}|p{5em}|p{5em}|p{7em}|p{5cm}| }
            \hhline{|=|=|=|=|=|}
            Nome & Formula & Derivata & Esempio & Nota \\
            
            \hline
            
            % funzioni costanti
            \begin{align*}
		    \text{Logaritmo}
            \end{align*} &
            \begin{align*}
                y=\log_a{x} 
            \end{align*}  &
            \begin{align*}
                y'=\frac{1}{x}\log_a{e} 
            \end{align*} &
            {
                \begin{align*}
                    y  &= \sqrt{x^4}    \\
                        &= x^\frac{4}{2}   \\
                    y' &= \frac{\cancel{4}^2}{\cancel{2}_1}x^{\frac{\cancel{4}^2}{\cancel{2}_1}-1}  \\
                        &= 2x^{2-1}  \\
                        &= 2x  
                \end{align*}
            } &
            {
            \begin{center}
                Se il logaritmo ha come base '\emph{e}' (quindi si tratta di logaritmo naturale), la formula si puo' contrarre. \\
                Esempio: \\
                \begin{align*}
                    y &= \ln{x}  \\
                    y' &= \frac{1}{x} 
                \end{align*}
            \end{center}
            } \\

            \hhline{|=|=|=|=|=|}
            
        \end{tabular}
    }
\end{center}

\end{document}
