\documentclass[../../main]{subfiles}
\begin{document}
\subsection{Funzioni Irrazionali}
\begin{center}
    \noindent\makebox[\textwidth]{
        \label{tab:derivate:irrazionali}
        \begin{tabular}{ |p{5em}|p{5em}|p{5em}|p{7em}|p{5cm}| }
            \hhline{|=|=|=|=|=|}
            Nome & Formula & Derivata & Esempio & Nota \\
            
            \hline
            
            % funzioni costanti
            \begin{align*}
		    \text{Radice quadrata}
            \end{align*} &
            \begin{align*}
                y=\sqrt{x} 
            \end{align*}  &
            \begin{align*}
                y'=\frac{1}{ 2\sqrt{x} } 
            \end{align*} &
            {
                \begin{align*}
                    y  &= \sqrt{x^4}    \\
                        &= x^\frac{4}{2}   \\
                    y' &= \frac{\cancel{4}^2}{\cancel{2}_1}x^{\frac{\cancel{4}^2}{\cancel{2}_1}-1}  \\
                        &= 2x^{2-1}  \\
                        &= 2x  
                \end{align*}
            } &
            \begin{center}
                La radice quadrata puo' essere espressa come potenza, quindi si puo' applicare la regola per le funzioni razionali intere.
            \end{center} \\ 

            \hhline{|=|=|=|=|=|}
            
        \end{tabular}
    }
\end{center}

\end{document}
