\section{Funzioni Logaritmiche}
\begin{center}
    \noindent\makebox[\textwidth]{
        \label{tab:derivate:logaritmiche}
        \begin{tabular}{ |p{5em}|p{5em}|p{5em}|p{7em}|p{5cm}| }
            \hline
            Nome & Formula & Derivata & Esempio & Nota \\
            
            \hline
            
            % funzioni costanti
            \begin{center}
                Logaritmo
            \end{center} &
            \begin{align}
                y=\log_a{x} \nonumber
            \end{align}  &
            \begin{align}
                y'=\frac{1}{x}\log_a{e} \nonumber
            \end{align} &
            {
                \begin{align}
                    y  &= \sqrt{x^4}   \nonumber \\
                        &= x^\frac{4}{2}  \nonumber \\
                    y' &= \frac{\cancel{4}^2}{\cancel{2}_1}x^{\frac{\cancel{4}^2}{\cancel{2}_1}-1} \nonumber \\
                        &= 2x^{2-1} \nonumber \\
                        &= 2x \nonumber 
                \end{align}
            } &
            {
            \begin{center}
                Se il logaritmo ha come base \emph{e} (quindi si tratta di logaritmo naturale), la formula si puo' contrarre. \\
                Esempio: \\
                \begin{align}
                    y &= \ln{x} \nonumber \\
                    y' &= \frac{1}{x} \nonumber
                \end{align}
            \end{center}
            } \\

            \hline
            
        \end{tabular}
    }
\end{center}