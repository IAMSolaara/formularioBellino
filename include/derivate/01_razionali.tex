\documentclass[../../main]{subfiles}
\begin{document}
\subsection{Funzioni Razionali}
\begin{center}
    \noindent\makebox[\textwidth]{
        \label{tab:derivate:razionali}
        \begin{tabular}{ |p{5em}|p{5em}|p{5em}|p{7em}|p{5cm}| }
            \hline
            Nome & Formula & Derivata & Esempio & Nota \\
            
            \hline
            
            % funzioni costanti
            \begin{center}
                Costante
            \end{center} &
            \begin{align}
                y=n \nonumber
            \end{align}  &
            \begin{align}
                y'=0 \nonumber
            \end{align} &
            {
                \begin{align}
                    y &= 4   \nonumber \\
                    y' & = 0 \nonumber 
                \end{align}
            } &
            \begin{center}
                La derivata di qualsiasi funzione costante equivale a 0 
            \end{center} \\ 

            \hline
            
            % razionali intere
            \begin{center}
                Intera
            \end{center} &
            \begin{align}
                y=x^n \nonumber
            \end{align} &
            \begin{align}
                y'=nx^{n-1} \nonumber
            \end{align} &
            {
                \begin{align}
                    y  & = 5x^2        \nonumber \\
                    y' & =(2*5)x^{2-1} \nonumber \\
                        & =10x          \nonumber
                \end{align}
            } &  \\

            \hline
            
            % razionali frazionarie
            \begin{center}
                Frazionaria
            \end{center} &
            \begin{align}
                y = \frac{n}{x} \nonumber
            \end{align} &
            \begin{center}
                ---
            \end{center} &
            \begin{center}
                ---
            \end{center} &
            \begin{center}
                Vedere "Rapporto" sezione \nameref{tab:derivate:operazioni} a pagina \pageref{tab:derivate:operazioni} \\
            \end{center}  \\
            \hline
        \end{tabular}
    }
\end{center}

\end{document}