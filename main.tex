\documentclass[10pt, a4paper]{article}
\usepackage[utf8]{inputenc}
\usepackage{mathtools}
\usepackage{array}
\usepackage[italian]{babel}
\usepackage[pdfusetitle]{hyperref}
\usepackage{cancel}
\usepackage{xcolor}
\usepackage[pdftex]{graphicx}
\usepackage{caption}
\usepackage{titling}
\usepackage{comment}

\title{Il formulario bellino}
\author{Lorenzo Cauli}

\setlength{\footskip}{90pt}
\setlength{\droptitle}{-10em}
\graphicspath{./images/}


\begin{document}
    \maketitle
    \hspace{-35pt}\includegraphics[width=400pt]{images/scano.png}
    \newpage
    \tableofcontents

    \newpage
    \section{Derivate}
    \documentclass[../../main]{subfiles}
\begin{document}
\subsection{Funzioni Razionali}
\begin{center}
    \noindent\makebox[\textwidth]{
        \label{tab:derivate:razionali}
        \begin{tabular}{ |p{5em}|p{5em}|p{5em}|p{7em}|p{5cm}| }
            \hhline{|=|=|=|=|=|}
            Nome & Formula & Derivata & Esempio & Nota \\
            
            \hline
            
            % funzioni costanti
            \begin{align*}
		    \text{Costante}
            \end{align*} &
            \begin{align*}
                y=n 
            \end{align*}  &
            \begin{align*}
                y'=0 
            \end{align*} &
            {
                \begin{align*}
                    y &= 4    \\
                    y' & = 0  
                \end{align*}
            } &
            \begin{center}
                La derivata di qualsiasi funzione costante equivale a 0 
            \end{center} \\ 

            \hline
            
            % razionali intere
            \begin{align*}
		    \text{Intera}
            \end{align*} &
            \begin{align*}
                y=x^n 
            \end{align*} &
            \begin{align*}
                y'=nx^{n-1} 
            \end{align*} &
            {
                \begin{align*}
                    y  & = 5x^2         \\
                    y' & =(2*5)x^{2-1}  \\
                        & =10x          
                \end{align*}
            } &  \\

            \hline
            
            % razionali frazionarie
            \begin{align*}
		    \text{Frazionaria}
            \end{align*} &
            \begin{align*}
                y = \frac{n}{x} 
            \end{align*} &
            \begin{center}
                ---
            \end{center} &
            \begin{center}
                ---
            \end{center} &
            \begin{center}
                Vedere "Rapporto" sezione \nameref{tab:derivate:operazioni} a pagina \pageref{tab:derivate:operazioni} \\
            \end{center}  \\
            \hhline{|=|=|=|=|=|}
        \end{tabular}
    }
\end{center}

\end{document}

    \documentclass[../../main]{subfiles}
\begin{document}
\subsection{Funzioni Irrazionali}
\begin{center}
    \noindent\makebox[\textwidth]{
        \label{tab:derivate:irrazionali}
        \begin{tabular}{ |p{5em}|p{5em}|p{5em}|p{7em}|p{5cm}| }
            \hhline{|=|=|=|=|=|}
            Nome & Formula & Derivata & Esempio & Nota \\
            
            \hline
            
            % funzioni costanti
            \begin{align*}
		    \text{Radice quadrata}
            \end{align*} &
            \begin{align*}
                y=\sqrt{x} 
            \end{align*}  &
            \begin{align*}
                y'=\frac{1}{ 2\sqrt{x} } 
            \end{align*} &
            {
                \begin{align*}
                    y  &= \sqrt{x^4}    \\
                        &= x^\frac{4}{2}   \\
                    y' &= \frac{\cancel{4}^2}{\cancel{2}_1}x^{\frac{\cancel{4}^2}{\cancel{2}_1}-1}  \\
                        &= 2x^{2-1}  \\
                        &= 2x  
                \end{align*}
            } &
            \begin{center}
                La radice quadrata puo' essere espressa come potenza, quindi si puo' applicare la regola per le funzioni razionali intere.
            \end{center} \\ 

            \hhline{|=|=|=|=|=|}
            
        \end{tabular}
    }
\end{center}

\end{document}

    \documentclass[../../main]{subfiles}
\begin{document}
\subsection{Funzioni Esponenziali}
\begin{center}
    \noindent\makebox[\textwidth]{
        \label{tab:derivate:esponenziali}
        \begin{tabular}{ |p{6em}|p{5em}|p{5em}|p{7em}|p{12em}| }
            \hhline{|=|=|=|=|=|}
            Nome & Formula & Derivata & Esempio & Nota \\
            
            \hline
            
            % funzioni costanti
            \begin{align*}
		    \text{Funzione esponenziale}
            \end{align*} &
            \begin{align*}
                y=n^x 
            \end{align*}  &
            \begin{align*}
                y'=n^xln(n) 
            \end{align*} &
            {
                \begin{align*}
                    y  &= \sqrt{x^4}    \\
                        &= x^\frac{4}{2}   \\
                    y' &= \frac{\cancel{4}^2}{\cancel{2}_1}x^{\frac{\cancel{4}^2}{\cancel{2}_1}-1}  \\
                        &= 2x^{2-1}  \\
                        &= 2x  
                \end{align*}
            } &
            \begin{center}
                La radice quadrata puo' essere espressa come potenza, \\
                quindi si puo' applicare la regola per le funzioni razionali intere.
            \end{center} \\ 

            \hhline{|=|=|=|=|=|}
            
        \end{tabular}
    }
\end{center}

\end{document}

    \documentclass[../../main]{subfiles}
\begin{document}
\subsection{Funzioni Logaritmiche}
\begin{center}
    \noindent\makebox[\textwidth]{
        \label{tab:derivate:logaritmiche}
        \begin{tabular}{ |p{5em}|p{5em}|p{5em}|p{7em}|p{5cm}| }
            \hhline{|=|=|=|=|=|}
            Nome & Formula & Derivata & Esempio & Nota \\
            
            \hline
            
            % funzioni costanti
            \begin{align*}
		    \text{Logaritmo}
            \end{align*} &
            \begin{align*}
                y=\log_a{x} 
            \end{align*}  &
            \begin{align*}
                y'=\frac{1}{x}\log_a{e} 
            \end{align*} &
            {
                \begin{align*}
                    y  &= \sqrt{x^4}    \\
                        &= x^\frac{4}{2}   \\
                    y' &= \frac{\cancel{4}^2}{\cancel{2}_1}x^{\frac{\cancel{4}^2}{\cancel{2}_1}-1}  \\
                        &= 2x^{2-1}  \\
                        &= 2x  
                \end{align*}
            } &
            {
            \begin{center}
                Se il logaritmo ha come base '\emph{e}' (quindi si tratta di logaritmo naturale), la formula si puo' contrarre. \\
                Esempio: \\
                \begin{align*}
                    y &= \ln{x}  \\
                    y' &= \frac{1}{x} 
                \end{align*}
            \end{center}
            } \\

            \hhline{|=|=|=|=|=|}
            
        \end{tabular}
    }
\end{center}

\end{document}

    \documentclass[../../main]{subfiles}
\begin{document}
\subsection{Funzioni Goniometriche}
\begin{center}
    \label{tab:derivate:goniometriche}
    \begin{longtable}{ |p{7em}|p{5em}|p{5em}|p{7em}|p{5cm}| }%
        \hhline{|=|=|=|=|=|}
        Nome & Formula & Derivata & Esempio & Nota \\
        \endfirsthead
        \hhline{|=|=|=|=|=|}
        \endlastfoot
        
        \hline
        
        % funzioni costanti
        \begin{align*}
            \text{Seno}
        \end{align*} &
        {
            \begin{align*}
                y=&\sin{x} 
            \end{align*}
        } &
        {
            \begin{align*}
                y'&=\cos{x} 
            \end{align*}
        } &
        {
            \begin{align*}
                y &= 4\sin{x}  \\
                y' &= 4\cos{x} 
            \end{align*}
        } &
        {
        \begin{center}
        \end{center}
        } \\

        \hline

        % funzioni costanti
        \begin{align*}
            \text{Coseno}
        \end{align*} &
        {
            \begin{align*}
                y&=\cos{x} 
            \end{align*}
        } &
        {
            \begin{align*}
                y'&=-\sin{x} 
            \end{align*}
        } &
        {
            \begin{align*}
                y &= 4\cos{x}  \\
                y' &= -4\sin{x} 
            \end{align*}
        } &
        {
        \begin{center}
        \end{center}
        } \\

        \hline

        % funzioni costanti
        \begin{align*}
            \text{Tangente}
        \end{align*} &
        {
            \begin{align*}
                y&=\tan{x} 
            \end{align*}
        } &
        {
            \begin{align*}
                y'&=\frac{1}{\cos^2{x}} 
            \end{align*}
        } &
        {
            \begin{align*}
                y  & = 4 \tan{x}  \\
                y' & = 4 \frac{1}{\cos^2{x}}  \\
                    & = \frac{4}{\cos^2{x}} 
            \end{align*}
        } &
        {
        \begin{center}
        \end{center}
        } \\

        \hline
        
        % funzioni costanti
        \begin{align*}
            \text{ArcoSeno}
        \end{align*} &
        {
            \begin{align*}
                y&=\arcsin{x} 
            \end{align*}
        } &
        {
            \begin{align*}
                y'&= \frac{1}{\sqrt{1-x^2}} 
            \end{align*}
        } &
        {
            \begin{align*}
                y & =\arcsin{0}  \\
                y' & = \frac{1}{\sqrt{1-(0)^2}}  \\
                    & = \frac{1}{\sqrt{1-0}}  \\
                    & = \frac{1}{1} = 1 
            \end{align*}
        } &
        {
        \begin{center}
        \end{center}
        } \\

        \hline

        % funzioni costanti
        \begin{align*}
            \text{ArcoCoseno}
        \end{align*} &
        {
            \begin{align*}
                y=&\arccos{x} 
            \end{align*} 
        } &
        {
            \begin{align*}
                y'=&-\frac{1}{\sqrt{1-x^2}} 
            \end{align*} 
        } &
        {
            \begin{align*}
                y & =\arcsin{0}  \\
                y' & = -\frac{1}{\sqrt{1-(0)^2}}  \\ 
                    & = -\frac{1}{\sqrt{1-0}}  \\
                    & = -\frac{1}{1} = -1 
            \end{align*}
        } &
        {
        \begin{center}
        \end{center}
        } \\

        \hline
        
        % funzioni costanti
        \begin{align*}
            \text{ArcoTangente} 
        \end{align*} &
        {
            \begin{align*}
                y=&\arctan{x} 
            \end{align*}  
        }&
        {
            \begin{align*}
                y'=&\frac{1}{1+x^2} 
            \end{align*} 
        }&
        {
            \begin{align*}
                y &= \frac{1}{2}\arctan{x}  \\
                y' &= \frac{1}{2} \frac{1}{1+x^2}  \\
                    &= \frac{1}{2(1+x^2)}  \\
                    &= \frac{1}{2+2x^2} 
            \end{align*}
        } &
        {
        \begin{center}
        \end{center}
        } \\

        \hline

        % funzioni costanti
        \begin{align*}
            \text{ArcoCotangente}
        \end{align*} &
        {
            \begin{align*}
                y=\arccot{x} 
            \end{align*}  
        }&
        {
            \begin{align*}
                y'= -\frac{1}{1+x^2} 
            \end{align*} 
        }&
        {
            \begin{align*}
                y &= \frac{1}{2}\arccot{x}  \\
                y' &= -\frac{1}{2} \frac{1}{1+x^2}  \\
                    &= -\frac{1}{2(1+x^2)}  \\
                    &= -\frac{1}{2+2x^2} 
            \end{align*}
        } &
        {
        \begin{center}
        \end{center}
        } \\

        \hline
        
        
    \end{longtable}
\end{center}

\end{document}
    \subsection{Operazioni tra funzioni}
\begin{center}
    \noindent\makebox[\textwidth]{
        \label{tab:derivate:operazioni}
        \begin{tabular}{ |p{5em} | p{5em} | p{5em} | p{7em} | p{2cm}| }
            \hline
            Nome & Formula & Derivata & Esempio & Nota \\
            
            \hline
            
            % somma funzioni
            \begin{center}
                Somma
            \end{center} &
            \begin{align}
                y= {\color{red}f(x)} + {\color{blue}g(x)} \nonumber
            \end{align}  &
            \begin{align}
                y'= {\color{red}f'(x)} + {\color{blue}g'(x)} \nonumber
            \end{align} &
            {
                \begin{align}
                    y  &= {\color{red}x^2} + {\color{blue}4}   \nonumber \\
                    y' &= {\color{red}2x^{2-1}} + {\color{blue}0} \nonumber \\
                        &= 2x \nonumber 
                \end{align}
            } &
            \begin{center}
            \end{center} \\ 

            \hline
            
            % differenza funzioni
            \begin{center}
                Differenza
            \end{center} &
            \begin{align}
                y= {\color{red}f(x)} - {\color{blue}g(x)} \nonumber
            \end{align}  &
            \begin{align}
                y'= {\color{red}f'(x)} - {\color{blue}g'(x)} \nonumber
            \end{align} &
            {
                \begin{align}
                    y  &= {\color{red}x^2} - {\color{blue}4}   \nonumber \\
                    y' &= {\color{red}2x^{2-1}} - {\color{blue}0} \nonumber \\
                        &= 2x \nonumber 
                \end{align}
            } &
            \begin{center}
            \end{center} \\ 

            \hline
            
            % prodotto funzioni
            \begin{center}
                Prodotto
            \end{center} &
            \begin{align}
                y= {\color{red}f(x)} * {\color{blue}g(x)} \nonumber
            \end{align}  &
            \begin{align}
                y'= ({\color{red}f'(x)} * {\color{blue}g(x)} ) + ( {\color{red}f(x)} * {\color{blue}g'(x)}) \nonumber
            \end{align} &
            {
                \begin{align}
                    y  &= {\color{red}x^2} * {\color{blue}4}   \nonumber \\
                    y' &= ( {\color{red}2x^{2-1}} * {\color{blue}4} ) + ( {\color{red}x^2} * {\color{blue}0} ) \nonumber \\
                        &= (2x * 4) + 0 \nonumber \\
                        &= 8x \nonumber 
                \end{align}
            } &
            \begin{center}
            \end{center} \\ 

            \hline
            
            % rapporto funzioni
            \begin{center}
                Rapporto
            \end{center} &
            \begin{align}
                y= \frac{\color{red}f(x)}{\color{blue}g(x)} \nonumber
            \end{align}  &
            \begin{align}
                y'= \frac{[{\color{red}f'(x)} * {\color{blue}g(x)} ] - [ {\color{red}f(x)} * {\color{blue}g'(x)}]}{[{\color{blue}g'(x)}]^2} \nonumber
            \end{align} &
            {
                \begin{align}
                    y  &= \frac{\color{red}x^2}{\color{blue}4}   \nonumber \\
                    y' &= \frac{( {\color{red}2x^{2-1}} * {\color{blue}4} ) - ( {\color{red}x^2} * {\color{blue}0} )}{{\color{blue}4}^2} \nonumber \\
                        &= \frac{({\color{red}2x} * {\color{blue}4}) - 0}{16} \nonumber \\
                        &= \frac{\cancel{8}^1x}{\cancel{16}_2} \nonumber  \\
                        &= \frac{x}{2} \nonumber
                \end{align}
            } &
            \begin{center}
            \end{center} \\ 

            \hline
            
            % funzioni composte
            \begin{center}
                Funzioni composte
            \end{center} &
            \begin{align}
                y= {\color{red}f({\color{blue}g(x)})} \nonumber
            \end{align}  &
            \begin{align}
                y'= {\color{red}f'({\color{blue}g(x)})} * {\color{blue}g'(x)} \nonumber
            \end{align} &
            {
                \begin{align}
                    y  &= {\color{red}sin({\color{blue}2x^2})}   \nonumber \\
                    y' &= {\color{red}cos({\color{blue}2x^2})} * {\color{blue}4x} \nonumber 
                \end{align}
            } &
            \begin{center}
            \end{center} \\ 

            \hline
            
        \end{tabular}
    }
\end{center}


    \newpage
    \section{Integrali}
    \subsection{AAAA}
    \huge{WIP}

\end{document} 