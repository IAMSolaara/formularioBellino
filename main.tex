\documentclass[10pt, a4paper]{article}
\usepackage[utf8]{inputenc}
\usepackage{mathtools}
\usepackage{array}
\usepackage[italian]{babel}
\usepackage{hyperref}

\title{Il formulario bellino}
\author{Lorenzo Cauli}

\begin{document}
    \maketitle
    \newpage
    \tableofcontents

    \newpage
    \part{Derivate}
    \section{Funzioni Razionali}
    \begin{center}
        \noindent\makebox[\textwidth]{
            \label{tab:derivate:razionali}
            \begin{tabular}{ |p{5em}|p{5em}|p{5em}|p{7em}|p{5cm}| }
                \hline
                Nome & Formula & Derivata & Esempio & Nota \\
                
                \hline
                
                % funzioni costanti
                \begin{center}
                    Costante
                \end{center} &
                \begin{align}
                    y=n \nonumber
                \end{align}  &
                \begin{align}
                    y'=0 \nonumber
                \end{align} &
                {
                    \begin{align}
                        y &= 4   \nonumber \\
                        y' & = 0 \nonumber 
                    \end{align}
                } &
                \begin{center}
                    La derivata di qualsiasi funzione costante equivale a 0 
                \end{center} \\ 

                \hline
                
                % razionali intere
                \begin{center}
                    Intera
                \end{center} &
                \begin{align}
                    y=x^n \nonumber
                \end{align} &
                \begin{align}
                    y'=nx^{n-1} \nonumber
                \end{align} &
                {
                    \begin{align}
                        y  & = 5x^2        \nonumber \\
                        y' & =(2*5)x^{2-1} \nonumber \\
                           & =10x          \nonumber
                    \end{align}
                } &  \\

                \hline
                
                % razionali frazionarie
                \begin{center}
                    Frazionaria
                \end{center} &
                \begin{align}
                    y = \frac{n}{x} \nonumber
                \end{align} &
                \begin{align}
                    y'=nx^{n-1} \nonumber
                \end{align} &
                \begin{center}
                    ---
                \end{center} &
                \begin{center}
                    Vedere sezione \nameref{eq:derivate:divisione} a pagina \pageref{eq:derivate:divisione} \\
                \end{center}  \\
                \hline
            \end{tabular}
        }
    \end{center}
    \section{Funzioni Irrazionali}
    \begin{center}
        \noindent\makebox[\textwidth]{
            \label{tab:derivate:razionali}
            \begin{tabular}{ |p{5em}|p{5em}|p{5em}|p{7em}|p{5cm}| }
                \hline
                Nome & Formula & Derivata & Esempio & Nota \\
                
                \hline
                
                % funzioni costanti
                \begin{center}
                    Radice quadrata
                \end{center} &
                \begin{align}
                    y=\sqrt{x} \nonumber
                \end{align}  &
                \begin{align}
                    y'=\frac{1}{ 2\sqrt{x} } \nonumber
                \end{align} &
                {
                    \begin{align}
                        y  &= \sqrt{x^4}   \nonumber \\
                           &= x^\frac{4}{2}  \nonumber \\
                        y' &= \frac{1}{ 2\sqrt{4} } \nonumber
                    \end{align}
                } &
                \begin{center}
                    La derivata di qualsiasi funzione costante equivale a 0 
                \end{center} \\ 

                \hline
                
            \end{tabular}
        }
    \end{center}
    \section{Funzioni Esponenziali}
    \section{Funzioni Logaritmiche}
    \section{Funzioni Goniometriche}

    \newpage
    \part{Integrali}
    \section{AAAA}
    \huge{WIP}

\end{document} 